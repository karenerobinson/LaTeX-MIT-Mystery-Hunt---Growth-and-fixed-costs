% assign each to a bank, give interest.
% write out descriptions (with highlighted clues.)
% add in inventories
% make cash flows all net to magic number at end of period.
% fix circular reference caused by cash asset (by using magic number!)
% hide letters:
% in income before taxes line
% in net cash flows from opex line
% in total current assets line
% in stockholders' equity (assets - liabilities)

\documentclass[10pt]{article}
\small

\usepackage{color}
\usepackage{latexsym,amssymb}
\usepackage{relsize,fancyhdr,multicol,array,multirow,amsmath}
\usepackage{tikz}
\usepackage{wordlike}
\usepackage{etoolbox}
\usepackage{siunitx} % allows aligning a tabular to a decimal point.

%\usepackage{forloop}

\usepackage{colortbl}

% create a column type to align at the decimal point.
\usepackage{dcolumn}
\newcolumntype{.}{D{.}{.}{-1}}

% end usepackage statements

%\addtolength{\oddsidemargin}{-.875in}
%\addtolength{\evensidemargin}{-.875in}
%\addtolength{\textwidth}{1.75in}

\addtolength{\topmargin}{-.5in}
\addtolength{\textheight}{1in}
\addtolength{\leftmargin}{-.5in}
\addtolength{\textwidth}{.5in}

\setlength{\parskip}{10pt plus 1pt minus 1pt}
\setlength{\parindent}{0pt}

% now some logic

\newtoggle{solution}
\toggletrue{solution}
\togglefalse{solution}

% set the colors for solution / puzzle:

\iftoggle{solution}{
\definecolor{soln-black}{gray}{0.0}
\definecolor{soln-lightblue}{rgb}{0.4,0.5,1}
}
{
\definecolor{soln-black}{gray}{1.0}
\definecolor{soln-lightblue}{rgb}{1,1,1}
}

% set definition that depend on colors:

\newcommand\mybox[2][]{\tikz[overlay]\node[fill=soln-lightblue!20,inner sep=2pt, anchor=text, rectangle, rounded corners=1mm,#1] {#2};\phantom{#2}}

% and now the document

\begin{document}

\iftoggle{solution}{

\setcounter{page}{0}

\section*{Growth and Fixed Costs: \textcolor{soln-lightblue}{Solution}}

\begin{tabular}{p{0.75\textwidth}!{\color{white}\vrule}p{0.25\textwidth}}

{\it The financial statements of two Wonderland banks and all their customers were set aside for tax irregularities... something here doesn't add up.}\\
\end{tabular}

\noindent\makebox[\linewidth]{\rule{\textwidth}{0.4pt}}

{\fontfamily{cmr}\selectfont
Things to notice about the financial statements:

\vspace{-.1in}
\begin{itemize}

\item There is enough information to match the company profiles to the
  financial statements, and so put the financial statements in order.

\item The number ``614050'' appears frequently and is in a different font.  This should be the most notable thing about the cash flow statements.

\item When the statements are put in order, the tax amounts (in the statement of operations) can be converted to numbers to read ``Taxes form a red herring.''

\item some lines in the Statement of Assets \& Liabilities don't add
  up right: they're off by a number between 1 and 26.

\end{itemize}
\vspace{-.1in}

Given the above, the steps to solving this puzzle are:

\vspace{.1in}
\textbf{\large 1. Match statements to company descriptions}

This gives an ordering of the financial statements.  

The keys to matching are:

\vspace{-.1in}
\begin{itemize}

\item Size of a company: there are a few initial revenue numbers: 100,000; 50,000; 30,000; 10,000.

\item Speed a company gets paid or pays its bills: 
``Receivables days'' and ``payables days'' are well-defined concepts.

\item Amount of inventory a company holds:\\
Restaurants, for instance, hold inventory only for a few days because most of it is perishable.

Note that the puzzle should be solvable without knowing those things
in advance, since ``payables days'', ``receivables days'', and
``inventory days'' are referred to in a company description, and these
are google-able.

\item interest paid on loans: higher by customers of First Wonderland, lower by customers of the Standard Bank.

\item ``Intangible assets'' or brand worth, held by companies with consumer brands

\item ``Goodwill'' some companies have accumulated by acquiring companies for more than the acquisitions' book values

\item And a few special items, such as:

\begin{itemize}
\vspace{-.1in}
\item legal fees awarded from one company to another

\item dividends paid by companies owned by the Red Queen or White King

\end{itemize}

\end{itemize}
\vspace{-.1in}

\vspace{.1in}
\textbf{\large 2. Find errors in the financial statements; convert to letters}

These are lines that don't ``add up''.  Each statement has a line off
by a value $<$ 26; these can be converted to letters (1 = A, 2 = B, etc) to
spell the clue phrase ``SEARCHHBSPHARVARDEDU'' or Search
hbsp.harvard.edu .

\vspace{.1in}
\textbf{\large 3. Search for the out-of-place number}

If you search hbsp.harvard.edu for ``13024''--- every company's
cash balance at the end of the period, the only number introduced
that's not a multiple of 100, and the amount of other items such as
legal fees or goodwill---you get a case with subtitle ``Women are the
solution'' (for testsolving) or with title ``The answer is secondary''
during hunt.

}}{}

\newpage


\iftoggle{solution}{
\section*{Growth and Fixed Costs: \textcolor{soln-lightblue}{Solution}}
}{
\section*{Growth and Fixed Costs}
}

\begin{tabular}{p{0.75\textwidth}!{\color{white}\vrule}p{0.25\textwidth}}

{\it The financial statements of two Wonderland banks and all their customers were set aside for tax irregularities... something here doesn't add up.}\\
\end{tabular}

\noindent\makebox[\linewidth]{\rule{\textwidth}{0.4pt}}


%% to do: 

% consider making the second parboxes multicolumn, and having columns
% for the different attributes on the right side.  This will also
% involve having space at the top (where the horizontal line is) for
% headings for the columns.

% consider adding lines between the highlighted bits of text and the
% reasons.  (though... why?)

\begin{tabular}{p{0.75\textwidth}!{\color{white}\vrule}p{0.25\textwidth}}


\parbox{0.72\textwidth}{
A commercial bank offering loans with \mybox{competitive interest rates} to Wonderland businesses; it has also \mybox{issued a consumer credit card}.
Walrus is a major shareholder.\\
} & 
\parbox{0.2\textwidth}{\iftoggle{solution}{\textcolor{soln-lightblue}{
lower interest payments\\
fee revenues\\
}
{}
}}

\\
\parbox{0.72\textwidth}{
A bank focused on small Wonderland merchants and loans for Wonderland individuals.  First Wonderland has \mybox{many branches in convenient areas} of Wonderland and \mybox{excellent customer service}.
} & 
\parbox{0.2\textwidth}{\iftoggle{solution}{\textcolor{soln-lightblue}{
higher occupancy costs\\
higher salary payments
}{}
}}

\\
\parbox{0.72\textwidth}{
A bank focused on small Wonderland merchants and loans for Wonderland individuals.  First Wonderland has \mybox{many branches in convenient areas} of Wonderland and \mybox{excellent customer service}.
} & 
\parbox{0.2\textwidth}{\iftoggle{solution}{\textcolor{soln-lightblue}{
higher occupancy costs\\
higher salary payments
}{}
}}

\\
\parbox{0.72\textwidth}{
Rust Works Steel \& Metal Productsis an industrial conglomerate that mines raw materials and makes steel and other industrial goods in a \mybox{highly-automated operation}.  It sells to Travels with Turtles
 as well as to Solid Manufacturing, Inc.
  The mines it owns are expected to be \mybox{in operation for close to a century}; other fixed assets are also long-lived, but are occasionally sold.\\

Rust Works Steel \& Metal Products is \mybox{slow to pay its bills} and \mybox{slow to collect payment} from its customers.  Similarly, its customers have high ``days payable'' and ``days receivable'' amounts.
} & 
\parbox{0.2\textwidth}{\iftoggle{solution}{\textcolor{soln-lightblue}{
low salary costs\\
long depreciation years (fixed assets / depr.)\\
\vspace{.1in}\\
high payables \& receivables days
}{}
}}

\\
\parbox{0.72\textwidth}{
A \mybox{popular, well-known} travel company that owns its own infrastructure and cars, each of which require replacement \mybox{every couple of decades}.  Travels with Turtles
 is financed with a loan from Standard Bank of Wonderland\hspace{-.04in}; \mybox{some years it loses money}.
} & 
\parbox{0.2\textwidth}{\iftoggle{solution}{\textcolor{soln-lightblue}{
Good brand recognition\\
20-yr depreciation\\
Deferred tax asset.
}{}
}}

\\
\parbox{0.72\textwidth}{
This innovative company is the premier producer of dyes and flavors in Wonderland:  Its ingredients are key for manufacture of blacking and whiting as well as toothpaste and any size-changing food or drink.  Wonderland D\&P \mybox{owns top-of-the-line} chemical manufacturing equipment.  It is known to \mybox{demand quick payment} from its customers, and this reputation helps its customers keep their accounts recievable balances low as well.
} & 
\parbox{0.2\textwidth}{\iftoggle{solution}{
\textcolor{soln-lightblue}{
high fixed assets \\
low accounts receivable
}{}
}}

\\
\parbox{0.72\textwidth}{
This contract manufacturer for Dovetail Design also makes tooling for White Rabbit Timepieces
and other customers.  It is heavily invested in \mybox{precision manufacturing equipment} and its margins are razor-thin.
} & 
\parbox{0.2\textwidth}{\iftoggle{solution}{
\textcolor{soln-lightblue}{
high depreciation, high fixed assets.\\
}{}
}}

\\
\parbox{0.72\textwidth}{
Leading provider of computational plug-ins for mechanical ``computers'' which are increasingly popular in Wonderland.  Dovetail offers modules that perform simple differentiation and integration that can be connected into simple parlour games; it continuously \mybox{develops new products}.  \\

Dovetail contracts with Solid Manufacturing, Inc.
for the high-precision manufacturing of its widgets.  Its \mybox{name is known} throughout Wonderland, and its products are sold in a variety of Wonderland shops and resellers.  Carpenter owns a majority stake and \mybox{refuses to bank with Standard Wonderland}.
} & 
\parbox{0.2\textwidth}{\iftoggle{solution}{\textcolor{soln-lightblue}{
R\&D expenditure\\
\vspace{.1in}\\
intangible brand asset\\
First Wonderland:\\
\hspace{.1in}(higher interest rate)
}{}
}}

\\
\parbox{0.72\textwidth}{
The Cheshire Cat's line of toothpaste.  The majority of ingredients are purchased from Wonderland Dyes and Potions
\hspace{-.08in}; its \mybox{materials are cheap} but its \mybox{advertising is expensive}, as is its \mybox{manufacturing equipment} which must be \mybox{replaced about every ten years}.  Like most customers of Wonderland Dyes and Potions
\hspace{-.04in}, Emblaze
 is \mybox{paid promptly and pays its bills promptly} as well.\\
} & 
\parbox{0.2\textwidth}{\iftoggle{solution}{\textcolor{soln-lightblue}{
good gross margin\\
high advertising cost\\
high fixed assets;\\
10-yr depreciation\\
low receivables days;\\
\hspace{.1in}low payables days
}{}
}}

\\
\parbox{0.72\textwidth}{
\mybox{One of many} manufacturers of blacking and whiting, this \mybox{small} company is dependent on the \mybox{only supplier for its key ingredients}, \mybox{Wonderland Dyes and Potions
}\hspace{-.04in}.   Owned by Gryphon, who prides himself on \mybox{managing cash flow} and keeping \mybox{inventory days low}.\\
} & 
\parbox{0.2\textwidth}{\iftoggle{solution}{
\textcolor{soln-lightblue}{
weak competitive position: low margin\\
low payables, rec'bles \& inventory\\
}{}
}}

\\
\parbox{0.72\textwidth}{
This independent watch manufacturer has a \mybox{strong brand} and strives to reach exclusive Wonderland customers. Many Wonderland shops carry one or two types of Rabbit watches; in contrast, the company store \mybox{carries each of hundreds} of models and sizes.  Distribution shops \mybox{pay Rabbit only after} a watch has sold.\\

The manufacturing is done by hand using specialized equipment that typically lasts about 15 years.
} & 
\parbox{0.2\textwidth}{\iftoggle{solution}{
\textcolor{soln-lightblue}{
Intangible assets\\
high receivables\\
\vspace{.1in}\\
High inventory
}{}
}}

\\
\parbox{0.72\textwidth}{
Large chain of customer shops owned by Caterpillar; these stores are the \mybox{only source} in Wonderland for some consumer goods such as Gryphon's Blacking and Whiting.  In the last few years Caterpillar \mybox{has acquired two smaller chains} of customer shops to move into higher foot-traffic locations; before the acquisitions Caterpillar Smoke and Gears
had several \mybox{money-losing years}.\\

Caterpillar Smoke and Gears
 \mybox{lost a trademark battle} last year to White's Chess Supply and Sundries
\hspace{-.04in}, and is \mybox{financed via loans} from Standard Bank of Wonderland\hspace{-.04in}.\\
} & 
\parbox{0.2\textwidth}{\iftoggle{solution}{
\textcolor{soln-lightblue}{
good gross margins.
\vspace{.12in}\\
goodwill asset.
\vspace{.12in}\\
deferred tax asset.\\
\vspace{.1in}\\
Legal fee;\\
debt-financed.  \\
}{}
}}

\\
\parbox{0.72\textwidth}{
Chain of consumer shops, owned by the court of the White King, that deals in many kinds of consumer goods and \mybox{funds upkeep} of the King's court.  Staff's activities include \mybox{development of new chess pieces} and use of new materials in boards and pieces.  
} & 
\parbox{0.2\textwidth}{\iftoggle{solution}{
\textcolor{soln-lightblue}{
pays dividends\\
R\&D 
}{}
}}

\\
\parbox{0.72\textwidth}{
This chain of shops has \mybox{an exclusive license} to sell size-changing Drinkme$^{TM}$ and Eatme$^{TM}$ products, which it makes using materials purchased from \mybox{Wonderland Dye}.  Scrumptious Essences
also \mybox{does some experimenting} on potions' effect on mirrors.
} & 
\parbox{0.2\textwidth}{\iftoggle{solution}{
\textcolor{soln-lightblue}{\\
high gross margins\\
low payables days\\
R\&D costs\\
}{}
}}

\\
\parbox{0.72\textwidth}{
This small customer shop, \mybox{like many in Wonderland}, sells looking glasses, toothpaste, and magic potions.  The brothers \mybox{own their building} and \mybox{staff the counter themselves} whenever possible.  The shop only keeps the most commonly needed items and is \mybox{often out-of-stock}.  Tweedle Brothers is \mybox{slow to pay its bills}.\\
} & 
\parbox{0.2\textwidth}{\iftoggle{solution}{\textcolor{soln-lightblue}{\vspace{-.3in}\\low margins\\
some depreciation\\
low salary costs\\
low inventory\\
high accounts payables.
}{}
}}

\\
\parbox{0.72\textwidth}{
Tea shop owned by the Mad Hatter, who spends some of his time \mybox{developing} new tea combinations.  Offers customers who are having an unbirthday the unusual option to pay after 30 days, but then often has \mybox{trouble collecting payment}.  

} & 
\parbox{0.2\textwidth}{\iftoggle{solution}{\textcolor{soln-lightblue}{
R\&D costs.\\
high receivables days\\
high gross margin\\
(from low tea costs)
}{}
}}

\\
\parbox{0.72\textwidth}{
Tea shop formerly part of Hatter's Tea-stop
\hspace{-.1in}, that split off a few years ago and subsequently \mybox{lost a suit} filed by the same.  Benefited from \mybox{low prices for raw tea across Wonderland} this year.  The corporate split was \mybox{funded by a loan} which \mybox{will come due} in a few years.
} & 
\parbox{0.2\textwidth}{\iftoggle{solution}{
\textcolor{soln-lightblue}{
one-time legal payment\\
high gross margin\\
debt-financed\\
}{}
}}

\\
\parbox{0.72\textwidth}{
Another sundries shop, one of the \mybox{best-respected names} among sundry shops in Wonderland. Offers guaranteed \mybox{availability of the widest range of goods} anywhere in Wonderland; suppliers grumble about Humpty Dumpty's fairly long \mybox{days payable}.
} & 
\parbox{0.2\textwidth}{\iftoggle{solution}{
\textcolor{soln-lightblue}{
intangible asset\\
high inventory\\
high payables\\
}{}
}}

\\
\parbox{0.72\textwidth}{
This oyster bar owned by Walrus is \mybox{not very popular and not very unpopular}.  Staff are known to be \mybox{grumpy}, but the oysters are always \mybox{fresh}.  The bar has borrowed heavily from Standard Bank of Wonderland\hspace{-.048in}, the smallest business to do so, and has \mybox{good relationships with vendors} of \mybox{lemon juice and wine}.\\

For years, the tax agency has puzzled over the Purple Walrus's low costs and low profitability; they wonder where the revenues are going.

%Walrus and the Red Queen are in a long-standing argument over the Wonderland flat tax rate of 30\%.  Walrus argues it is too high; the Queen returns that her businesses pay it too, so clearly the rate is fair.
} &
\parbox{0.2\textwidth}{\iftoggle{solution}{
\textcolor{soln-lightblue}{
smaller intangible asset\\
staff are underpaid;\\
inventory is low\\
payables not too high\\
No oyster vendors...\\
\\
``banking fees'' are hiding revenue
}{}
}}

\\
\parbox{0.72\textwidth}{
A small pub owned by a group of the Red Queen's guards that operates \mybox{out of a house} in the court and receives food and beer deliveries \mybox{daily}.  The guards are happy waiters and bartenders, even though they expect the Queen to take any of the pub's profits.\\

Right Raven is financed via a \mybox{large loan from First Wonderland} that they expect to roll over \mybox{when it comes due} in several years. \\
} & 
\parbox{0.2\textwidth}{\iftoggle{solution}{
\textcolor{soln-lightblue}{
no rental costs\\
inventory $<$ 3 days'\\
profits are kept low; salaries are high\\
\\
high long-term debt; no current portion\\
}{}
}}

\\

\end{tabular}



\foreach \n in {1,...,20}{
      \begin{tabular}{p{0.75\textwidth}!{\color{white}\vrule}p{0.25\textwidth}}

      \textbf{{\n}. \input{\n}}\\
      \input{desc-\n}\\
      \end{tabular}

}


% \iftoggle{solution}{
%     \foreach \n in {1,...,20}{
%       \begin{tabular}{p{0.4\textwidth} | c  | c  | c  | c  | c  | c  | c  | c  | c  | c | c  | c  | c  |}
%       \textbf{{\n}.} \input{\n} & \hspace{10pt}  & \hspace{10pt}  & \hspace{10pt}  & \hspace{10pt}  & \hspace{10pt}  & \hspace{10pt}  & \hspace{10pt}  & \hspace{10pt}  & \hspace{10pt}  & \hspace{10pt}  & \hspace{10pt} & \hspace{10pt} & \hspace{10pt}\\
%       \hline
%       \end{tabular}\\
%     } % close foreach    
% }{}

\foreach \n in {12,19,3,6,16,8,11,18,9,2,17,7,1,13,20,5,15,4,10,14}{
%\foreach \n in {1,...,20}{
    \newpage

    \iftoggle{solution}{
      \begin{tabular}{p{0.75\textwidth}!{\color{white}\vrule}p{0.25\textwidth}}

      \textbf{{\n}. \input{\n}}\\
      \input{desc-\n}
      \end{tabular}

    }{
      \vspace{2in}
    }
    \input{statement-\n}

    \input{soln-\n}


  }%


\end{document}
